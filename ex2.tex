\section{Heating and cooling in HII regions}

\subsection{}
HII regions are in equilibrium, but there are still a number of processes ongoing.
Two of these are photoionization and radiative recombination.
As one is a heating process and the other a cooling, they can keep each other in equilibrium.
This is the case when the heating rate is equal to the cooling rate.
We can find the temperature at which this occurs by minimizing the absolute value of the difference between the heating and cooling rates.
As the photoionization rate is given by:

\begin{align}
    \Gamma_{pe} = \alpha_B n_H n_e \psi k_B T_c
\end{align}
where $\alpha_B$ is the recombination coefficient, $n_H$ is the number density of hydrogen, $n_e$ is the number density of electrons, 
$k_B$ is the Boltzmann constant, and $T_c$ is the temperature of the gas.
\\
and the radiative recombination rate is given by:

\begin{align}
    \Lambda_{rr} = \alpha_B n_H n_e \left< E_{rr} \right>
\end{align}

with:

\begin{align}
    \left< E_{rr} \right> = \left[ 0.684 - 0.0416 ln\left( \frac{T_4}{Z^2} \right) \right] k_B T
\end{align}
where $T_4$ is the temperature in units of $10^4$ K and $Z$ is the metallicity (which we assume to be the solar metallicity $Z=0.015$).

These combine to:
\begin{align}
    \Delta E_{net} = \psi T_c k_B - \left[ 0.684 - 0.0416 * ln\left( \frac{T_4}{Z^2} \right) \right] T k_B
\end{align}

By minimizing $| \Delta E_{net} |$ using Brent's method we find an equilibrium temperature of:
\lstinputlisting{output/2a.txt}

code used for this subsection:
\lstinputlisting{ex2.py}[language=Python,firstline=1,lastline=210]

\subsection{}
Only including photoionization and radiative recombination is not a very realistic configuration, though.
Therefore we will now also include heating by cosmic rays ($\Gamma_{CR}$) and MHD waves ($\Gamma_{MHD}$) and cooling by free-free emission ($\Lambda_FF$).
These follow the expressions:
\begin{align} 
    \Lambda_{FF} &= 0.54 T^{0.37}_4 \alpha_B n_e n_H k_B T \\
    \Gamma_{CR} &= A n_e \xi_{CR} \\
    \Gamma_{MHD} &= 8.9\dot10^{-26} n_H T_4\\
\end{align}

Combining these with photoionization and radiative recombination, we get:
\begin{align}
    \Delta E_{net} = \left[ \psi T_c - \left[ 0.684 - 0.0416 * ln\left( \frac{T_4}{Z^2} \right)\right] T - 0.54 * T_4^{0.37} T\right] k_B n_H \alpha_B + A \xi + 8.9\dot10^{-26} T_4
\end{align}

By minimizing $| \Delta E_{net} |$ using Brent's method we find an equilibrium temperature of:
\lstinputlisting{output/2b.txt}

Here we see that the equilibrium temperature decreases for higher electron densities.
This can be explained by noticing that $\Gamma_{pe}$, $\Lambda_{rr}$, and $\Lambda_{FF}$ all depend on $n_e^2$ 
(in HII regions most of the gas is ionized, therefore $n_e \approx n_H$), while $\Gamma_{CR}$ and $\Gamma_{MHD}$ only depend on $n_e$.
As $\Gamma_{CR}$ and $\Gamma_{MHD}$ are both heating processes, the cooling rate will increase faster with $n_e$ than the heating rate.
This leads to the equilibrium temperature decreasing with increasing electron density.

code used for this subsection:
\lstinputlisting{ex2.py}[language=Python,firstline=212]