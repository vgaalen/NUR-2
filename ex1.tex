\section{Satellite Galaxies Around a Missive Central}

\subsection{}
Write a numberical integrator based on some form of Richardson
extrapolation to solve equation (2) for A given those four parameters. Output A to full precision.
Hint: First think about what dV is for a spherical integral!
\\
\begin{align}
    \int \int \int _V n(x) dV = \left< N_{sat} \right>\\
    \left< N_{sat} \right> &= \int \int \int _V A \left< N_{sat} \right> \left( \frac{x}{b} \right)^{a-3} exp\left[-\left(\frac{x}{b}\right)^c\right] dV \\
    1 / A &= \int \int \int _V \left( \frac{x}{b} \right)^{a-3} exp\left[-\left(\frac{x}{b}\right)^c\right] dV \\
    &= \int_0^{r_{max}} \int_0^{2 \pi} \int_0^{\pi} \left( \frac{x}{b} \right)^{a-3} exp\left[-\left(\frac{x}{b}\right)^c\right] r^2 sin(\phi) dr d\theta d\phi \\
    &= \int_0^{r_{max}} 4 \pi r^2 \left( \frac{r}{b} \right)^{a-3} exp\left[-\left(\frac{r}{b}\right)^c\right] dr \\
    &= \int_0^{r_{max}} 4 \pi r^{a-1} b^{3-a} exp\left[-\left(\frac{r}{b}\right)^c\right] dr \\
\end{align}

Solving this using Romburg integration, we get the following value for A:
\lstinputlisting{output/1a.txt}
\noindent
Code used for this subquestion:
\lstinputlisting{ex1.py}[firstline=1, lastline=56]


\subsection{}
Here I used a XOR shift RNG feeding into a MLCG RNG to generate random numbers between 0 and 1.
From this RNG I took 30.000 samples to use for random positions for 10.000 satellite galaxies in 3 dimensions.
Transforming the uniform distribution to the given density profile difficult as the CDF of the satellite number density is not easily invertible.
Therefore rejection sampling was used here.
Figure \ref{fig:1b} shows that the distribution of satellite galaxies is in agreement with the analytical function.
\begin{figure}[H]
    \centering
    \includegraphics[width=0.8\textwidth]{plots/my_solution_1b.png}
    \caption{The distribution of satellite galaxies compared to the analytical function.}
    \label{fig:1b}
\end{figure}


As the number density is only a function of distance to the central galaxy, this is only necessary for the radial dimension.
As a uniform distribution over all points with the same r, is not the same as a uniform distribution along dimensions $\phi$ and $\theta$ we need still need to transform
the distribution along at least one of these dimensions.
For $\theta$ we will choose a uniform distribution: $\theta = 2 \pi U(0,1)$ (where $U(0,1)$ represents a RNG sampling on a uniform distribution).
As $p(\theta,\phi)d\theta d\phi = \frac{sin \phi}{4 \pi} d\phi d\theta$,
this choice results in $p(\phi) = \frac{sin \phi}{2}$.
The resulting CDF is $P(\phi) = \int_0^{\phi} \frac{sin \phi}{2} d\phi = \frac{1-cos \phi}{2}$.
As this function is easily invertible we will use the transformation method here:
$P^{-1}(\phi) = arccos(1-2\phi)$.
As shown in figure \ref{fig:phi} this results in the desired sinusoidal distribution.
\begin{figure}[H]
    \centering
    \includegraphics[width=0.8\textwidth]{plots/phi.png}
    \caption{The distribution the phi coordinates compared to the desired distribution.}
    \label{fig:phi}
\end{figure}

For completeness (and as a pure test of the RNG) we will check if the distribution of the $\theta$ coordinates is uniform.
Figure \ref{fig:theta} shows the distibution along with a line at $\frac{N_{generated}}{N_{bins}}$.
This is not a real test of randomness, but does snow that we get a uniform distribution.
\begin{figure}[H]
    \centering
    \includegraphics[width=0.8\textwidth]{plots/theta.png}
    \caption{The distribution the theta coordinates compared to the desired distribution.}
    \label{fig:theta}
\end{figure}

Code used for this subquestion:
\lstinputlisting{ex1.py}[firstline=60, lastline=179]

\subsection{}
In order to adhere to all three of the rules

\begin{enumerate}
    \item select every galaxy with equal probability;
    \item not draw the same galaxy twice;
    \item not reject any draw.
\end{enumerate}

the simplest approach seems to shuffle the list of galaxies and then draw the first 100.
Here I shuffled the list using a set of random numbers generated by the RNG from the previous question, which I sorted and during this sorting the same entry swaps
were applied to the list of galaxies.
As it is not necessary here to sort the whole list, I opted to use selection sort as this algorithm allows you to sort only the first 100 entries with minimal changes.

Figure \ref{fig:1c} shows the cumulative number of galaxies over the relative radius for this subset. When we compare this to figure \ref{fig:1b} we can conclude that
they follow the same distribution, which tells us that the subsample taken can draw every galaxy with equal probability.
\begin{figure}
    \centering
    \includegraphics[width=0.8\textwidth]{plots/my_solution_1c.png}
    \caption{The Cumulative number of galaxies over the relative radius using the subsample of 100 galaxies.}
    \label{fig:1c}
\end{figure}

Code used for this subquestion:
\lstinputlisting{ex1.py}[firstline=183, lastline=220]

\subsection{}
Here I calculated the numerical derivative using an implementation of Ridder's Method.
Below the output of this is compared to the following analytical derivative:
\begin{align}
    \\frac{d n}{d x} = \fraq{A N_sat}{b} \left( \frac{x}{b} \right)^(a-4) e^{-(x/b)^c} \left[ -c \left( \frac{x}{b} \right)^c + a-3 \right]
\end{align}

\lstinputlisting{output/1d.txt}
This method seems to work quite well as I was able the deviation from the analytical solution down to the order of $10^{-12}$.

Code used for this subquestion:
\lstinputlisting{ex1.py}[firstline=224]